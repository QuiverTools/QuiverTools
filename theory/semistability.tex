\documentclass[a4paper]{article}
\pdfoutput=1
\usepackage{hyperref}
\hypersetup{hypertexnames = false, bookmarksdepth = 2, bookmarksopen = true, colorlinks, linkcolor = black, citecolor = black, urlcolor = black, pdfstartview={XYZ null null 1}}

\usepackage{amsfonts}
\usepackage[fleqn, leqno]{amsmath}
\usepackage{amsthm}

\usepackage[capitalise]{cleveref}

% as per https://tex.stackexchange.com/questions/306908/switching-from-amsrefs-to-biblatex
\begin{filecontents*}{mrnumber.dbx}
\DeclareDatamodelFields[type=field,datatype=verbatim]{mrnumber}
\DeclareDatamodelEntryfields{mrnumber}
\end{filecontents*}
\usepackage[backend=biber, maxbibnames=99, datamodel=mrnumber, sortcites]{biblatex}

\usepackage{booktabs}
\usepackage{colonequals}
\usepackage{diagbox}
\usepackage{enumitem}
\usepackage{mathtools}
\usepackage{parskip}
\usepackage{thmtools}
\usepackage{tikz-cd}
\usepackage[colorinlistoftodos, textsize = footnotesize]{todonotes}
\usepackage{xparse}
\usepackage{xspace}

\usepackage[utf8]{inputenc}
\usepackage[T1]{fontenc}
\usepackage{libertine}
\usepackage[libertine]{newtxmath}
\usepackage[scaled=0.83]{beramono}
%\usepackage[charter]{mathdesign}
%\usepackage[scaled]{beramono,berasans}
\usepackage{eucal}
\usepackage{microtype}
\frenchspacing

\usepackage{gitinfo2}

% bibliography configuration
\DeclareFieldFormat{mrnumber}{%
  MR\addcolon\space
  \ifhyperref
    {\href{http://www.ams.org/mathscinet-getitem?mr=1#1}{\nolinkurl{#1}}}
    {\nolinkurl{#1}}}

\renewbibmacro*{doi+eprint+url}{%
  \iftoggle{bbx:doi}
    {\printfield{doi}}
    {}%
  \newunit\newblock
  \printfield{mrnumber}%
  \newunit\newblock
  \iftoggle{bbx:eprint}
    {\usebibmacro{eprint}}
    {}%
  \newunit\newblock
  \iftoggle{bbx:url}
    {\usebibmacro{url+urldate}}
    {}}

\addbibresource{automatic.bib}
\addbibresource{other.bib}

% mathematics configuration
\relpenalty=10000
\binoppenalty=10000

% todonotes configuration
\newcounter{todocounter}
\DeclareDocumentCommand\addreference{g}{\stepcounter{todocounter}\todo[color = blue!30]{\thetodocounter. Add reference\IfNoValueF{#1}{: #1}}\xspace}
\DeclareDocumentCommand\checkthis{g}{\stepcounter{todocounter}\todo[color = red!50]{\thetodocounter. Check this\IfNoValueF{#1}{: #1}}\xspace}
\DeclareDocumentCommand\fixthis{g}{\stepcounter{todocounter}\todo[color = orange!50]{\thetodocounter. Fix this\IfNoValueF{#1}{: #1}}\xspace}
\DeclareDocumentCommand\expand{g}{\stepcounter{todocounter}\todo[color = green!50]{\thetodocounter. Expand\IfNoValueF{#1}{: #1}}\xspace}
\newcommand\removethis{\stepcounter{todocounter}\todo[color=yellow!50]{\thetodocounter. Remove this?}}

% environments
\declaretheoremstyle[
  spaceabove = 3pt,
  spacebelow = 3pt,
  bodyfont = \itshape,
]{first}
\declaretheoremstyle[
  spaceabove = 3pt,
  spacebelow = 3pt,
]{second}
\declaretheorem[numberwithin=section, style=first]{theorem}
\declaretheorem[sibling=theorem, style=first]{conjecture}
\declaretheorem[sibling=theorem, style=first]{corollary}
\declaretheorem[sibling=theorem, style=first]{lemma}
\declaretheorem[sibling=theorem, style=first]{proposition}

\declaretheorem[sibling=theorem, style=second]{example}
\declaretheorem[sibling=theorem, style=second]{remark}
\declaretheorem[sibling=theorem, style=second]{definition}
\declaretheorem[sibling=theorem, style=second]{notation}

\declaretheorem[numberwithin=section, style=first, title=Theorem]{alphatheorem}
\declaretheorem[sibling=alphatheorem, style=first, title=Conjecture]{alphaconjecture}
\declaretheorem[sibling=alphatheorem, style=first, title=Corollary]{alphacorollary}
\declaretheorem[sibling=alphatheorem, style=first, title=Proposition]{alphaproposition}

\renewcommand{\thealphatheorem}{\Alph{alphatheorem}}
\renewcommand{\thealphaconjecture}{\Alph{alphatheorem}}
\renewcommand{\thealphacorollary}{\Alph{alphatheorem}}
\renewcommand{\thealphaproposition}{\Alph{alphaproposition}}
\crefname{alphatheorem}{Theorem}{Theorems}
\crefname{alphaconjecture}{Conjecture}{Conjectures}
\crefname{alphacorollary}{Corollary}{Corollaries}
\crefname{alphaproposition}{Proposition}{Propositions}

\makeatletter
\def\gitfootnote{\gdef\@thefnmark{}\@footnotetext}
\makeatother

\mathchardef\mhyphen="2D
\newcommand\dash{\nobreakdash-\hspace{0pt}}

\DeclareMathOperator\Hom{Hom}
\DeclareMathOperator\Ext{Ext}

\newcommand\HN{\ensuremath{\mathrm{HN}}}

\newcommand{\Z}{\ensuremath{\mathbb{Z}}}

% macro for moduli stack resp. space of representations of a quiver
\DeclareDocumentCommand\modulistack{om}{\IfNoValueTF{#1}{\mathcal{M}{(#2)}}{\mathcal{M}^{#1}(#2)}}
\DeclareDocumentCommand\modulispace{om}{\IfNoValueTF{#1}{\mathrm{M}{(#2)}}{\mathrm{M}^{#1}(#2)}}
\DeclareDocumentCommand\representationspace{om}{\IfNoValueTF{#1}{\mathrm{R}{(#2)}}{\mathrm{R}^{#1}(#2)}}

% semistability, stability
\newcommand\sstable[1]{\ensuremath{{#1}\mhyphen\mathrm{sst}}}
\newcommand\stable[1]{\ensuremath{{#1}\mhyphen\mathrm{st}}}

\DeclareMathOperator\GL{\mathrm{GL}}



\title{An equivalent charcterization of semistability}
\author{}

\begin{document}
\maketitle

We work over an algebraically closed field. Let $Q$ be a quiver, $\mathbf{d}$ a dimension vector and $\theta \in \Z^{Q_0}$ a stability parameter. The slope function $\mu = \mu_\theta$ is defined by
\[
    \mu(\mathbf{e}) = \frac{\theta\cdot \mathbf{e}}{\sum_i e_i}    
\]
for every dimension vector $\mathbf{e} \neq 0$.

\begin{proposition}
    The following are equivalent:
    \begin{enumerate}
        \item There exists a $\theta$-semistable representation of dimension vector $\mathbf{d}$.
        \item For all sequences $\mathbf{d}^* = (\mathbf{d}^1,\ldots,\mathbf{d}^\ell)$ of length $\ell > 1$ with $\mu(\mathbf{d}^1) > \ldots > \mu(\mathbf{d}^\ell)$ holds
        \[
            \sum_{1 \leq r < s \leq \ell} \langle \mathbf{d}^r, \mathbf{d}^s \rangle < 0.    
        \]
    \end{enumerate}
\end{proposition}

\begin{proof}
    We prove (1) implies (2). Let $M$ be a $\theta$-semistable representation of dimension vector $\mathbf{d}$. Assume that there exists a sequence $\mathbf{d}^*$ of length $\ell > 1$ such that the slopes are strictly decreasing and such that $\sum_{r < s} \langle \mathbf{d}^r,\mathbf{d}^s \rangle \geq 0$. Consider the variety
    \[
        X = \prod_{i \in Q_0} \Fl_{d_i^*}(M_i)    
    \]
    of tuples $(F_i^r)$ of flags
    \[
        0 = F_i^0 \subseteq F_i^1 \subseteq \ldots \subseteq F_i^l = M_i     
    \]
    such that the subquotients $F_i^r/F_i^{r-1}$ have dimension $d_i^r$ for all $i \in Q_0$ and $r = 1,\ldots,\ell$. Inside $X$ we define the quiver flag variety $\Fl_\mathbf{e}(M)$ as the closed subset of all tuples of flags such that 
    \[
        M_a(F_{\source(a)}^r) \subseteq F_{\target(a)}^r     
    \]
    for all $a \in Q_1$ and for all $r=1,\ldots,\ell-1$. This subset has a natural structure of a closed subscheme of $X$ as follows. On $X$, we consider the universal bundles $\mathcal{F}_i^r$ for $i \in Q_0$ and $r=1,\ldots,\ell$; the fiber of $\mathcal{F}_j^s$ in a point $x = (F_i^r)$ is the vector space $F_j^s$. Let $a \in Q_1$. Consider the morphism
    \[
        \mathcal{F}_{\source(a)}^r \to \mathcal{O}_X \otimes M_{\source(a)} \to \mathcal{O}_X \otimes M_{\target(a)} \to (\mathcal{O}_X \otimes M_{\target(a)})/\mathcal{F}_{\target(a)}^r     
    \]
    given by the composition of the inclusion, the linear map $M_a$ and the projection. We obtain a global section $s_M$  of 
    \[
        \bigoplus_{a \in Q_1} \bigoplus_{r=1}^{\ell-1} \mathcal{F}_{\source(a)}^r \otimes (\mathcal{O}_X \otimes M_{\target(a)})/\mathcal{F}_{\target(a)}^r
    \]
    whose zero locus is $\Fl_\mathbf{e}(M)$. The scheme structure of $\Fl_\mathbf{e}(M)$ is by definition the scheme structure of $Z(s_M)$. It may be non-reduced, whence the term ``quiver flag variety'' is maybe not great, but it is standard terminology.
\end{proof}

\gitfootnote{commit: \texttt{\gitAbbrevHash}\hfil date: \texttt{\gitAuthorIsoDate}\hfil \texttt{\gitReferences}}

\end{document}
